\documentclass[12pt]{article}

\usepackage[ngerman]{babel}
\usepackage[utf8]{inputenc}
\usepackage[scale=0.80]{geometry}
\usepackage{amsmath}
\usepackage{amssymb}
\usepackage{graphicx}
\usepackage{fancyhdr}

\renewcommand{\familydefault}{\sfdefault}
\renewcommand{\arraystretch}{1.25}
\setlength{\headheight}{28pt}
\pagestyle{fancy}

\lhead{\textbf{Datenbanksysteme II}\\
\textbf{Lösung von Aufgabe 2}}
\rhead{\textbf{Felix Wolff, Markus Petrykowski}\\
\textbf{Übungsgruppe B}}
\cfoot{}
\rfoot{Seite \thepage}
\renewcommand{\footrulewidth}{0pt}

\begin{document}
Anzahl Blöcke die für n Datensätze gebraucht werden:
\begin{align*}
    \lceil \frac{n}{(0,8 * 30)} \rceil
\end{align*}

\noindent
Da in jeden Block nur 80\% der 30 Datensätze passen, können nur 24 Datensätze in einen Block. Das aufrunden ist notwendig, da wir am Ende einen Block haben, der zu weniger als 80 \% gefüllt sein kann, wenn n nicht genau aufgeht. \\

\noindent
\textbf{a)} 
Anzahl Blöcke für den dicht besetzten index:
\begin{align*}
    &\lceil \frac{n}{(0,8 * 200)} \rceil \\
	\Rightarrow
	&\lceil \frac{n}{(0,8 * 30)} \rceil +  \lceil \frac{n}{(0,8 * 200)} \rceil
\end{align*}

\noindent 
Bei einem dicht besetzten Index gibt es für jeden Wert in einem Block einen Index.  Daher Werden alle Datensätze auf die 160 Datensätze verteilt. 

\noindent
\textbf{b)} 
Anzahl Blöcke für den dünn besetzten Index:
\begin{align*}
    &\lceil \frac{\lceil \frac{n}{(0,8 * 30)} \rceil}{(0,8 * 200)} \rceil \\
	\Rightarrow
	&\lceil \frac{n}{(0,8 * 30)} \rceil +  \lceil \frac{\lceil \frac{n}{(0,8 * 30)} \rceil}{(0,8 * 200)} \rceil
\end{align*}

\noindent
Der dicht besetzte Index braucht nur für jeden Block der durch die Datensätze entsteht einen Indexeintrag.
\end{document}
% vim: tw=80

\end{document}