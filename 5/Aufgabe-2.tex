\documentclass[12pt]{article}

\usepackage[ngerman]{babel}
\usepackage[utf8]{inputenc}
\usepackage[scale=0.80]{geometry}
\usepackage{amsmath}
\usepackage{amssymb}
\usepackage{graphicx}
\usepackage{fancyhdr}
\usepackage{enumerate}

\renewcommand{\familydefault}{\sfdefault}
\renewcommand{\arraystretch}{1.25}
\setlength{\headheight}{28pt}
\pagestyle{fancy}

\lhead{\textbf{Datenbanksysteme II}\\
\textbf{Lösung von Aufgabe 2}}
\rhead{\textbf{Felix Wolff, Markus Petrykowski}\\
\textbf{Übungsgruppe B}}
\renewcommand{\footrulewidth}{0pt}

\begin{document}
Aus der Tabelle mit den Wertverteilungen ergeben sich $T(R)=52$ und $T(S)=78$.
Die einfache Abschätzung ergibt damit
$$T (R \bowtie S) = \frac{52*78}{20} = 202,8 \approx = 203$$

Bei der komplexeren Schätzung nutzt man die Werte aus der Tabelle um die
Join-Kardinalitäten für die Werte $0$,$1$ und $2$ zu berechnen:
$5*10+6*8+4*5 = 118$. Da keine genauen Werte für die Vorkommen von $3$ und $4$ in
beiden Relationen vorhanden sind, muss man jeweils für die andere Relation
schätzen: $5*\frac{48}{V(S,B)-4} = 15$ und $7*\frac{32}{V(R,B)-4} = 14$. Für die
restlichen Werte, für die keine Histogramm-Daten existieren, wird $(V(R,B)-5) *
3 * 2$ geschätzt.

Insgesamt ergibt sich die Schätzung mit den Histogramm-Daten zu:
$$T(R \bowtie S) = 118 + 15 + 14 + 15*3*2 = 237$$
\end{document}
% vim: tw=80
