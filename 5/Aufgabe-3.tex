\documentclass[12pt]{article}

\usepackage[ngerman]{babel}
\usepackage[utf8]{inputenc}
\usepackage[scale=0.80]{geometry}
\usepackage{amsmath}
\usepackage{amssymb}
\usepackage{graphicx}
\usepackage{fancyhdr}
\usepackage{enumerate}

\renewcommand{\familydefault}{\sfdefault}
\renewcommand{\arraystretch}{1.25}
\setlength{\headheight}{28pt}
\pagestyle{fancy}

\lhead{\textbf{Datenbanksysteme II}\\
\textbf{Lösung von Aufgabe 3}}
\rhead{\textbf{Felix Wolff, Markus Petrykowski}\\
\textbf{Übungsgruppe B}}
\renewcommand{\footrulewidth}{0pt}

\begin{document}
\textbf{Erste Runde}

\begin{center}
\begin{tabular}{c|c|c|c|c|c|c}
    Menge & ${E,F}$ & ${E,G}$ & ${E,H}$ & ${F,G}$ & ${F,H}$ & ${G,H}$\\
    Kardinalität & 20 & 10 & 800 & 240 & 200 & 80\\
    Kosten & 0 & 0 & 0 & 0 & 0 & 0\\
    Reihenfolge & $E\bowtie F$ & $E\bowtie G$ & $E\bowtie H$ & $F\bowtie G$ & $F\bowtie H$ & $G\bowtie H$\\
\end{tabular}
\end{center}

\textbf{Zweite Runde:} Dem Prinzip der Optimalität folgend wird nur der beste Teilplan weiter beobachtet- also $E \bowtie G$.
Dabei werden wegen \textit{Containment of Value Sets} die jeweiligen kleinsten Distinct-Werte der Joinergebnisse für die Rechnung verwendet. Also z.B. $V(E \bowtie G, A) = 50 = min(1000,50)$.
\begin{center}
\begin{tabular}{c|c|c}
    Menge & ${E,G,F}$ & ${E,G,H}$\\
    Kardinalität & $\frac{4}{500}$ & $\frac{8}{500}$ \\
    Kosten & 10 & 10\\
    Reihenfolge & $(E\bowtie G)\bowtie F$ & $(E\bowtie G)\bowtie H$\\
\end{tabular}
\end{center}

\textbf{Dritte Runde:}  Es kann nur einen geben (unter der gleichen Betrachtung wie in der zweiten Runde)
\begin{center}
\begin{tabular}{c|c}
    Menge & ${E,G,F,H}$ \\
    Kardinalität & $\frac{1}{62500}$ \\
    Kosten & 10 \\
    Reihenfolge & $((E\bowtie G)\bowtie F) \bowtie H$ \\
\end{tabular}
\end{center}
\end{document}
% vim: tw=80