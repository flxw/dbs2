\documentclass[12pt]{article}

\usepackage[ngerman]{babel}
\usepackage[utf8]{inputenc}
\usepackage[scale=0.80]{geometry}
\usepackage{amsmath}
\usepackage{amssymb}
\usepackage{graphicx}
\usepackage{fancyhdr}

\renewcommand{\familydefault}{\sfdefault}
\renewcommand{\arraystretch}{1.25}
\setlength{\headheight}{28pt}
\pagestyle{fancy}

\lhead{\textbf{Datenbanksysteme II}\\
\textbf{Lösung von Aufgabe 1}}
\rhead{\textbf{Felix Wolff, Markus Petrykowski}\\
\textbf{Übungsgruppe B}}
\cfoot{}
\rfoot{Seite \thepage}
\renewcommand{\footrulewidth}{0pt}

\begin{document}

\begin{tabular}{c|cc|c}
Query \# & Normal Zeit & optimierte Zeit & Ersparnis in \%	\\
\hline
1	&	31,117 s	&	25,423 s&	18,299\\
2	& 	3,282 s	    &	0,731 s	&	77,727\\
3	&	15,959 s	&	10,74 s	&	32,703\\
4	&	9,799 s	    &	3,534 s	&	63,935\\
5	&	37,902 s	&	3,582 s	&	90,549\\
6	&	9,831 s	    &	9,651 s	&	1,831\\
7	&	12,733 s	&	10 s	&	21,464\\
8	&	49,642 s	&	1,59 s	&	69,797\\
9	&	23,588 s	&	9,727 s	&	58,763\\
10	&	22,433 s	& 	11,536 s&	48,576\\
11	&	1,579 s	    &	0,511 s	&	67,638\\
12	&	14,046 s	&	10,758 s&	23,409\\
13	&	4,261 s	    &	1,46 s	&	65,736\\
14	&	9,784 s	    &	9,810 s	&	-0,266\\
15	&	9,602 s	    &	7,321 s	&	23,755\\
16	&	1,090 s	    &	1,21 s	&	-11,009\\
17	&	3,185 s	    &	2,282 s	&	28,352\\
18	&	9,183 s	    &	4,772 s	&	48,034\\
19	&	26,141 s	&	9,423 s	&	63,953\\
20	&	4,527 s	    &	3,927 s	&	13,254\\
21	&	39,763 s	&	6,457 s	&	83,761\\
22	&	3,006 s	    &	0,493 s	&	83,599\\
\hline
Total&	342,453 s	&	144,938 s	&	57,677\\
\end{tabular}

\newpage
\section*{Maßnahmen}

\subsubsection*{Maßnahmen für Query 1}
Zuerst werden detaillierte Statistiken für das Attribut \verb=L_SHIPDATE=
gesammelt. Diese helfen bei der Aufwandsabschätzung, da mittels diesem Attribut
selektiert wird. Zusätzlich wird noch ein Index auf \verb=L_LINESTATUS,
L_RETURNFLAG= erstellt. Dieser ist hilfreich, da dies die Gruppierungsattribute
sind.

\subsubsection*{Maßnahmen für Query 2}
Hier haben wir die von dbadvis vorgeschlagenen Maßnahmen angewandt, also Indizes
erstellt, die den Join beschleunigen, da sie alle Daten beinhalten, die der Join
benötigt. Dies bedeutet, dass keine Daten als Teile von Tupeln gelesen werden,
die dann wieder verworfen werden, weil sie nicht im Ergebnis enthalten sein
sollen.

\subsubsection*{Maßnahmen für Query 3}
Hier haben wir einen Index auf \verb=C_MKTSEGMENT, C_CUSTKEY= erstellt, der
den Join, in dem die beiden Attribute enthalten sind, vereinfacht.

\subsubsection*{Maßnahmen für Query 4}
Um die Subanfrage zu beschleunigen, haben wir einen Index auf \verb=L_ORDERKEY,
L_COMMITDATE, L_RECEIPTDATE= erstellt.

\subsubsection*{Maßnahmen für Query 5, 7, 11, 18, 21}
Hier hat uns dbadvis veschiedene Indizes vorgeschlagen, die einfach alle
Join-Attribute enthalten, und damit einfach die gelesene Datenmenge
reduzieren.

\subsection{Alle anderen}
Für alle anderen Queries haben wir keine Änderungen vorgenommen, diese haben von
den zahlreichen erstellten Indizes profitiert, und auch deutlich an
Geschwindigkeit gewonnen.

\end{document}
