\documentclass[12pt]{article}

\usepackage[ngerman]{babel}
\usepackage[utf8]{inputenc}
\usepackage[scale=0.80]{geometry}
\usepackage{amsmath}
\usepackage{amssymb}
\usepackage{graphicx}
\usepackage{fancyhdr}
\usepackage{enumerate}
\usepackage{enumitem}

\renewcommand{\familydefault}{\sfdefault}
\renewcommand{\arraystretch}{1.25}
\setlength{\headheight}{28pt}
\pagestyle{fancy}

\lhead{\textbf{Datenbanksysteme II}\\
\textbf{Lösung von Aufgabe 4}}
\rhead{\textbf{Felix Wolff, Markus Petrykowski}\\
\textbf{Übungsgruppe B}}
\renewcommand{\footrulewidth}{0pt}

\begin{document}
Es wird angenommen, dass die richtige Anfrage in relationaler Algebra folgendermaßen aussieht:
$$\sigma_{titel='King Kong'}(SpieltMit) \bowtie \sigma_{geschlecht='w'}(Schauspieler)$$
Weiterhin gehen wir davon aus, dass ein Block genau ein Tupel fasst, sodass
$B(R) = T(R)$.
\newline
\begin{itemize}[leftmargin=5em]
    \item[\textbf{Sort-basiert:}] Beim einfachen Sort-basierten Join
        werden beide Relationen erst mit TPMMS vorsortiert (Hier könnte dann auch
        gleich die Selektion angewandt werden). Dieser erste schritt kostet
        schon 4 I/O pro Block. Anschließend werden die Köpfe der sortierten
        Relationen gelesen und die passenden Tupel verjoint. Somit ergeben sich
        hier I/O-Kosten von $5*(B(SpieltMit)+B(Schauspieler)) = 100000$.

    \item[\textbf{Index-basiert:}] Zuerst wird der Index auf
        \textit{Schauspieler.name} eingelesen. Dann werden die Tupel von
        \textit{SpieltMit} sequentiell eingelesen, und immer geprüft, ob sich
        der Name des aktuellen SpieltMit-Tupels im Index von Schauspieler
        wiederfindet. Wenn ja, wird dieses Tupel geholt, und auf
        \textit{geschlecht='w'} geprüft. Trifft dies auch zu, wird gejoint.
        Angenommen, im Schnitt gäbe es
        $\frac{B(Schauspieler)}{V(Schauspieler,Y)}$ gleichnamige Schauspieler
        $Y$ (sagen wir, jeder 100te heißt gleich). Muss der Index nicht noch extra von der Disk geholt werden,
        so ergeben sich IO-Kosten von
        $B(SpieltMit)*\frac{B(Schauspieler)}{V(Schauspieler,Y)} = 10000 * \frac{10000}{100} = 1000000$.

    \item[\textbf{Hash-basiert:}] Alle Tupel von R und S werden jeweils
        eingelesen, nach Join-Attributen in Buckets gehasht, und diese Buckets
        wieder auf Disk geschrieben. Anschließend werden die Buckets wieder
        gelesen, woraus sich $3*(B(S)*B(R)) = 3*(10000+10000) = 60000$ I/O
        ergeben.
\end{itemize}
\end{document}
% vim: tw=80
