\documentclass[12pt]{article}

\usepackage[ngerman]{babel}
\usepackage[utf8]{inputenc}
\usepackage[scale=0.80]{geometry}
\usepackage{amsmath}
\usepackage{amssymb}
\usepackage{graphicx}
\usepackage{fancyhdr}
\usepackage{enumerate}

\renewcommand{\familydefault}{\sfdefault}
\renewcommand{\arraystretch}{1.25}
\setlength{\headheight}{28pt}
\pagestyle{fancy}

\lhead{\textbf{Datenbanksysteme II}\\
\textbf{Lösung von Aufgabe 2}}
\rhead{\textbf{Felix Wolff, Markus Petrykowski}\\
\textbf{Übungsgruppe B}}
\renewcommand{\footrulewidth}{0pt}

\begin{document}
\begin{enumerate}[a)]
    \item In der ersten Phase wird immer der ganze zur Verfügung stehende
        Hauptspeicher befüllt. Also entstehen bei der Sortierung von R
        $\frac{2^{13}}{2^7} = 2^6 = 64$ Teillisten. Bei der Sortierung von S
        sind es $\frac{2^{10}}{2^7} = 2^3 = 8$ Teillisten.

    \item In Phase 1 werden bis zu $M - 1$ sortierte Teillisten erzeugt. In
        Phase 2 werden denn die Köpfe dieser Liste in den Hauptspeicher geladen.
        Ein Block ist für den Output reserviert. Dementsprechend können maximal
        $(M-1) * M$ Blöcke sortiert werden, hier also $128*127 = 16256 \approx
        2^{17}$

    \item Beim Sort-Merge-Join werden erst die beiden Relationen R und S mittel
        TPMMS vorsortiert. Dies kostet pro Block 4 I/O, da die Teillisten
        erstellt, geschrieben, deren Köpfe gelesen und sortiert werden müssen,
        und dann das Endergebnis geschrieben werden muss. Also $4*(B(R) + B(S))
        I/O$ für Phase 1. In der zweiten Phase werden die Köpfe der sortierten
        Relationen gelesen und im Falle eines gemeinsamen Tupels $A$ werden die
        Tupel verjoint, was insgesamt auch wieder $B(R)+B(S) I/O$ kostet.
        Insgesamt fallen also $5*(B(R)+B(S)) = 41600 I/O$ an.

    \item Beim TPMMS wird eine Teilliste mit nur einem Block darin angelegt.
        Dies ist aber kein Problem in dem Sinne, dass dadurch das Verfahren
        nicht mehr funktioniert.
\end{enumerate}
\end{document}
% vim: tw=80
