\documentclass[12pt]{article}

\usepackage[ngerman]{babel}
\usepackage[utf8]{inputenc}
\usepackage[scale=0.80]{geometry}
\usepackage{amsmath}
\usepackage{amssymb}
\usepackage{graphicx}
\usepackage{fancyhdr}

\renewcommand{\familydefault}{\sfdefault}
\renewcommand{\arraystretch}{1.25}
\setlength{\headheight}{28pt}
\pagestyle{fancy}

\lhead{\textbf{Datenbanksysteme II}\\
\textbf{Lösung von Aufgabe 1}}
\rhead{\textbf{Felix Wolff, Markus Petrykowski}\\
\textbf{Übungsgruppe B}}
\cfoot{}
\rfoot{Seite \thepage}
\renewcommand{\footrulewidth}{0pt}

\begin{document}
\noindent Annahme der Paramter:

x	-	Anzahl der Pointer/Value Paare eines B+-Baum Index \\
\indent y 	-	Anzahl der Pointer/Value Paare eine Hash-Indexes \\


\noindent \textbf{a)} \\
\textbf{Scannen der kompletten Relation:}
\begin{align*}
	1000000 / 10 = 100000 
\end{align*}

\noindent \textbf{Nutzen eines B+-Baum Indexes für R.a:}
\begin{align*}
	Tiefe des Baums:  \log_{x}100000 \\
\end{align*}
Für x = 100 beträgt die Tiefe 3. Somit 100000 + 10000 + 1= 110001 Block gelesen werden. \\


\noindent \textbf{Nutzen eines Hash-Indexes für R.a}
\begin{align*}
	Anzahl Index Bloecke: 100000 / y   
\end{align*}

\noindent
Für y = 100 gibt es 1000 Blöcke.
Dementsprechend müssen 100000+1000 Blöcke eingelesen werden\\

\noindent \textbf{b)} \\
\textbf{Scannen der kompletten Relation:}
\begin{align*}
	1000000 / 10 = 100000 
\end{align*}

\noindent \textbf{Nutzen eines B+-Baum Indexes für R.a:}
\begin{align*}
	Tiefe des Baums:  \log_{x}100000 \\
\end{align*}
Für x = 100 beträgt die Tiefe 3. Folglich müssen 3 + 5  =  8 Blöcke gelesen werden.\\


\noindent \textbf{Nutzen eines Hash-Indexes für R.a}
\begin{align*}
	Anzahl Index Bloecke: 100000 / y   
\end{align*}

\noindent
Für y = 100 gibt es 1000 Blöcke. Es müssen potentiell 50 + 50 = 100 verschiedene Blöcke eingelesen werden

\newpage
\noindent \textbf{c)} \\
\textbf{Scannen der kompletten Relation:}
\begin{align*}
	1000000 / 10 = 100000 
\end{align*}

\noindent \textbf{Nutzen eines B+-Baum Indexes für R.a:}
\begin{align*}
	Tiefe des Baums:  \log_{x}100000 \\
\end{align*}
Für x = 100 beträgt die Tiefe 3. Es müssen also 3+1= 4 Blöcke gelesen werden\\


\noindent \textbf{Nutzen eines Hash-Indexes für R.a: }
\begin{align*}
	Anzahl Index Bloecke: 100000 / y   
\end{align*}

\noindent
Für y = 100 gibt es 1000 Blöcke.
Es müssen genau 2 Blöcke eingelesen werden.\\


\noindent \textbf{d)} \\
\textbf{Scannen der kompletten Relation:}
\begin{align*}
	1000000 / 10 = 100000 
\end{align*}

\noindent \textbf{Nutzen eines B+-Baum Indexes für R.a:}
\begin{align*}
	Tiefe des Baums:  \log_{x}100000 \\
\end{align*}
Für x = 100 beträgt die Tiefe 3. Folglich müssen 3 + 5   = 8 Blöcke gelesen werden.\\


\noindent \textbf{Nutzen eines Hash-Indexes für R.a:}
\begin{align*}
	Anzahl Index Bloecke: 100000 / y   
\end{align*}

\noindent Für y = 100 gibt es 1000 Blöcke.
Es müssen potentiell 50 + 50 verschiedene Blöcke eingelesen werden


\end{document}