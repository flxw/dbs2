\documentclass[12pt]{article}

\usepackage[ngerman]{babel}
\usepackage[utf8]{inputenc}
\usepackage[scale=0.80]{geometry}
\usepackage{amsmath}
\usepackage{amssymb}
\usepackage{graphicx}
\usepackage{fancyhdr}

\renewcommand{\familydefault}{\sfdefault}
\renewcommand{\arraystretch}{1.25}
\setlength{\headheight}{28pt}
\pagestyle{fancy}

\lhead{\textbf{Datenbanksysteme II}\\
\textbf{Lösung von Aufgabe 1}}
\rhead{\textbf{Felix Wolff, Markus Petrykowski}\\
\textbf{Übungsgruppe B}}
\cfoot{}
\rfoot{Seite \thepage}
\renewcommand{\footrulewidth}{0pt}

\begin{document}
\noindent Annahme der Paramter:

x	-	Anzahl der Pointer/Value Paare eines B+-Baum-Index-Blocks \\
\indent y 	-	Anzahl der Pointer/Value Paare eine Hash-Index-Blocks. Wir
nehmen an, dass ein Block gleichzeitig einem Bucket entspricht. \\


\noindent \textbf{a)} \\
\textbf{Scannen der kompletten Relation:}
\begin{align*}
	1000000 / 10 = 100000  IO
\end{align*}

\noindent \textbf{Nutzen eines B+-Baum Indexes für R.a:}
\begin{align*}
	Tiefe des Baums:  \log_{x}100000 \\
\end{align*}
Für x = 100 beträgt die Tiefe 3.
%Somit 100000 + 10000 + 1= 110001 Block gelesen werden. \\
Durch Traversion von der Wurzel bis zum links-äußersten Blatt werden 2 IO
benötigt. Da die Blätter verlinkt sind, können sie der Reihe nach gelesen
werden, sodass $\frac{1000000}{x}=10000$ IO gebraucht werden. Für jedes Blatt
werden die verlinkten Datenblöcke der Relation geholt, sofern sie noch nicht
gelesen wurden. Insgesamt also $2 + 10000 + 100000 = 110002 IO$.\\

\noindent \textbf{Nutzen eines Hash-Indexes für R.a}
Anzahl der Indexblöcke: $\frac{100000}{y}$.
Für y = 100 umfasst der Index 1000 Blöcke.
Dementsprechend müssen $100000+1000 = 101000$ Blöcke eingelesen werden\\

\noindent \textbf{b)} \\
\textbf{Scannen der kompletten Relation:}
\begin{align*}
	1000000 / 10 = 100000 
\end{align*}

\noindent \textbf{Nutzen eines B+-Baum Indexes für R.a:}
\begin{align*}
	Tiefe des Baums:  \log_{x}100000 \\
\end{align*}

Für x = 100 beträgt die Tiefe 3. Folglich müssen $3 + 5  =  8$ Blöcke gelesen werden.
Dies gilt dann, wenn jeweils 10 der gesuchten Tupel gemeinsam auf einem Block
liegen\\

\noindent \textbf{Nutzen eines Hash-Indexes für R.a}
\begin{align*}
	Anzahl Index Bloecke: 100000 / y   
\end{align*}

\noindent
Für $y = 100$ gibt es 1000 Blöcke. Es muss ein Block des Index gelesen werden,
da auf diesem alle Werte bis 100 indiziert sind. Es müssen potentiell $1 + 5 = 6$ verschiedene Blöcke eingelesen werden

\newpage
\noindent \textbf{c)} \\
\textbf{Scannen der kompletten Relation:}
\begin{align*}
	1000000 / 10 = 100000 
\end{align*}

\noindent \textbf{Nutzen eines B+-Baum Indexes für R.a:}
\begin{align*}
	Tiefe des Baums:  \log_{x}100000 \\
\end{align*}
Für x = 100 beträgt die Tiefe 3. Es müssen also $3+1=4$ Blöcke gelesen werden.\\


\noindent \textbf{Nutzen eines Hash-Indexes für R.a: }
Es müssen genau 2 Blöcke eingelesen werden, der Bucketblock, und der Datenblock.\\


\noindent \textbf{d)} \\
\textbf{Scannen der kompletten Relation:}
\begin{align*}
	1000000 / 10 = 100000 
\end{align*}

\noindent \textbf{Nutzen eines B+-Baum Indexes für R.a:}
\begin{align*}
	Tiefe des Baums:  \log_{x}100000 \\
\end{align*}
Von der Wurzel bis einschließlich zum ersten Blatt mit Werten jenseits der 50 sind es 3 IO.
Dieses eine Blatt ist wegen $x=100$ vollkommen ausreichend. Folglich müssen
$3+5=8$ Blöcke gelesen werden, wenn die gesuchten Tupel günstig auf einem
gemeinsamen Block liegen.

\noindent \textbf{Nutzen eines Hash-Indexes für R.a:}
\begin{align*}
	Anzahl Index Bloecke: 100000 / y   
\end{align*}

\noindent Für y = 100 gibt es 1000 Blöcke. Idealweise sind die Pointer für die
Werte 51 bis 99 auf einem Indexblock vorhanden (1 IO), und die korrespondierenden Tupel
gemeinsam auf einem Block (5 IO). Somit müssten $1 + 5 = 6$ IO aufgewandt werden.

\end{document}
