\documentclass[12pt]{article}

\usepackage[ngerman]{babel}
\usepackage[utf8]{inputenc}
\usepackage[scale=0.80]{geometry}
\usepackage{amsmath}
\usepackage{amssymb}
\usepackage{graphicx}
\usepackage{fancyhdr}

\renewcommand{\familydefault}{\sfdefault}
\renewcommand{\arraystretch}{1.25}
\setlength{\headheight}{28pt}
\pagestyle{fancy}

\lhead{\textbf{Datenbanksysteme II}\\
\textbf{Lösung von Aufgabe 2}}
\rhead{\textbf{Felix Wolff, Markus Petrykowski}\\
\textbf{Übungsgruppe B}}
\cfoot{}
\rfoot{Seite \thepage}
\renewcommand{\footrulewidth}{0pt}

\begin{document}
\noindent\textbf{a)} 
Während der ersten Phase wird der zur Verfügung stehende Hauptspeicher so oft voll
geladen, bis alle Tupel in sortierten Teillisten auf die Platte geschrieben
wurden. Der Hauptspeicher fasst 100MiB, ein Block hat 2KiB.
Damit passen $100MiB / 2KiB = 102400KiB / 2KiB = 51200$ Blöcke, also $51200 * 100
= 5120000$ Tupel in den Hauptspeicher.
\\[.5em]
Bei 10 Millionen Tupeln sind das also zwei Hauptspeicherfüllungen.
Wenn wir davon ausgehen, dass die optimale Verteilung auf die Zylinder die
Anordnung der Blöcke nebeneinander beinhaltet, sind $5120000/2500 = 2048$ Spuren
für eine Hauptspeicherfüllung zu lesen.
\\[.5em]
Das Lesen einer Hauptspeicherfüllung benötigt also $2048 * L_{avgTrack} = 2048 * 13.332ms = 27303ms$,
das Schreiben genau gleich lange. Daraus folgt, dass Phase 1 beim Schreiben und
Lesen insgesamt $2*2*27303ms = 109215.74$ benötigt.
\\[.5em]
Phase zwei wiederholt dies, indem die Köpfe der sortierten Teillisten gelesen
werden, sortiert werden, und wieder geschrieben werden. Also
benötigt der TPMMS insgesamt $2*2*2*27303ms = 218431.488ms$.
\\[1em]
\textbf{b)} Aus der Vorlesung gilt die Näherung, dass mit $m$ Byte
Hauptspeicher,einer Blockgröße von $b$ Byte und einer Tupelgröße von $r$ Byte
maximal $\frac{m^2}{rb}$ Tupel sortiert werden können.
Für die Rechnung ist also die gegebene Anzahl von 10 Millionen Tupel das
Maximum, während $m$ die Unbekannte darstellt. Mit $b=4*512 Byte = 2048 Byte$ und
$r = 20Byte$
\begin{align*}
    10M &= \frac{m^2}{2048 * 20}\\
    \Leftrightarrow m^2 &= 10M * 2048 * 20\\
    m^2 &= 409600M byte^2\\
    \sqrt{m^2} &= 640000 byte = 640KB
\end{align*}
\end{document}
% vim: tw=80
