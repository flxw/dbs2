\documentclass[12pt]{article}

\usepackage[ngerman]{babel}
\usepackage[utf8]{inputenc}
\usepackage[scale=0.80]{geometry}
\usepackage{amsmath}
\usepackage{amssymb}
\usepackage{graphicx}
\usepackage{fancyhdr}

\renewcommand{\familydefault}{\sfdefault}
\renewcommand{\arraystretch}{1.25}
\setlength{\headheight}{28pt}
\pagestyle{fancy}

\lhead{\textbf{Datenbanksysteme II}\\
\textbf{Lösung von Aufgabe 4}}
\rhead{\textbf{Felix Wolff, Markus Petrykowski}\\
\textbf{Übungsgruppe B}}
\cfoot{}
\rfoot{Seite \thepage}
\renewcommand{\footrulewidth}{0pt}

\begin{document}
\textbf{a)} 
Die Wahrscheinlichkeit $p$ besagt, wie wahrscheinlich jedes der optionalen
Attribute verwendet wird.  Also ist die erwartete Größe des Tupels in
Abhängigkeit von $p$ bei Datensätzen fester Länge (jedes Extra-Attribut benötigt
10 Byte):
$$40 Byte + 25 * p * 10 Byte$$

Die Sequenz der optionalen Attribute wird immer mit einem 2-Byte-Tag
abgeschlossen, sodass selbst bei keinem optionalen Attribut der Datensatz 42
Byte groß ist. Ansonsten gilt:
$$40 Byte + 2 Byte + 25*p*2 Byte$$

\textbf{b)}
Trägt man beide Funktionen abhängig von $p$ auf einem Intervall von $[0;1]$ auf,
so fällt auf dass die Speicherung mit Datensätzen fester Größe bis zu einer
Wahrscheinlichkeit von $p=0.0125$ platzsparender ist. Dies liegt darin
begründet, dass bei keinem optionalen Tupel die 2 Byte des Schlusstags zu
unnötigem Speicherverbrauch führen.
\end{document}
% vim: tw=80
