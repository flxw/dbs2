\documentclass[12pt]{article}

\usepackage[ngerman]{babel}
\usepackage[utf8]{inputenc}
\usepackage[scale=0.80]{geometry}
\usepackage{amsmath}
\usepackage{amssymb}
\usepackage{graphicx}
\usepackage{fancyhdr}
\usepackage{gensymb}

\renewcommand{\familydefault}{\sfdefault}
\renewcommand{\arraystretch}{1.25}
\setlength{\headheight}{28pt}
\pagestyle{fancy}

\lhead{\textbf{Datenbanksysteme II}\\
\textbf{Lösung von Aufgabe 4}}
\rhead{\textbf{Felix Wolff, Markus Petrykowski}\\
\textbf{Übungsgruppe B}}
\renewcommand{\footrulewidth}{0pt}

\begin{document}
\textbf{a)}  Ein Hersteller Block enthält 11 Tupel, ein Produkte Block 13 Tupel.
\newline
25000 / 11 =  2272 + 8 Tupel a' 350 Byte \newline
75000 / 13 =  5769 + 3 Tupel a' 300 Byte \newline
Da die Tupel von Produkten und Herstellern nicht in einem Block sein sollen, können die "Resttupel" nicht zusammen gelegt werden, folglich werden diese Jeweils in einen eigenen Block gelegt.
\newline
Insgesamt werden also 5769 + 2272 + 2 = 8043 Blöcke benötigt 

\textbf{b)}  Jeweils 3 Hersteller passen zusammen mit ihren Produkten in einen Block (3750 Byte) 
25000 / 3 = 8333,332 \^= 8334 Blöcken

\textbf{c)}  1) Speichervariante a, da einfach die zusammenhängenden Blöcke der Hersteller ausgelesen werden 	müssen. Vorteilhaft an Variante a ist, dass aller Hersteller am Stück ausgelesen werden können, ohne dass dafür große Strecken vom Lesekopf abgefahren werden müssen.

2) Variante b, da Hersteller und Produkte bereits zusammen liegen und man nicht für jeden Hersteller erst andere Blöcke zum finden der relevanten Daten ansteuern muss. Für den Join, müssen also keine Vergleiche mehr durchgeführt werden, da die zusammengehörigen Tupel schon zusammen liegen.

\end{document}
% vim: tw=80
